\documentclass[12pt,a4paper]{article}
\usepackage{indentfirst}
\usepackage[utf8x]{inputenc}
\usepackage{ucs}
\usepackage[MeX]{polski}
\usepackage{fancyhdr}
\usepackage{amsmath}
\usepackage{amsfonts}
\usepackage{amssymb}
\usepackage{subfig}
%\usepackage{supertabular}
\usepackage{array}
\usepackage{tabularx}
\usepackage{hhline}
\usepackage{tabulary, booktabs}
\usepackage{float}
\usepackage{lscape}
\usepackage[table]{xcolor}
\pagestyle{fancy}
\usepackage{graphicx}
\usepackage{multirow}
\newenvironment{bottompar}{\par\vspace*{\fill}}{\clearpage}

\begin{document}

\clearpage
\thispagestyle{empty}

\begin{figure}[H]
\centering
\includegraphics[scale=1.3]{logo.png}
\end{figure}

\vspace{16pt}

\begin{center}
\textbf{\huge Bazy Danych (Projekt)}

\vspace{30pt}

\textbf{\LARGE Analiza danych pogodowych}


\vspace{22pt}

\LARGE Opis projektu

\end{center}

\vspace{20pt}

\flushleft Autorzy: Dymitr Choroszczak 218627, Krzysztof Dąbek 218549\\
Kurs: Bazy danych (projekt)\\
Temat: Analiza danych\\
Prowadzący: dr hab. inż. Grzegorz Mzyk, prof. PWr\\
Termin zajęć: piątek 9:15\\


\newpage

\tableofcontents

\newpage

\section{Opis projektu}

\subsection{Koncept projektu}
\normalsize
Projekt jest realizowany w ramach kursu Bazy Danych na specjalności Robotyka (ARR), na kierunku Automatyka i Robotyka (AiR), na wydziale Elektroniki (EKA), na Politechnice Wrocławskiej.\\
Celem projektu jest stworzenie bazy danych przechowującej informacje pogodowe z różnych stacji, pobierane z serwera oraz zaimplementowanie metod analizy danych i wizualizacji wyników.\\
Dane do analizy zostaną pobrane z serwera ogimet.com dla wielu stacji pogodowych w Polsce i zapisane w tabelach.\\
Standardowy użytkownik bazy ma uprawnienia jedynie do wyświetlania wyników analizy oraz danych pogodowych.\\
Administrator ma dodatkowo możliwość dodawania, usuwania i zmiany stacji pogodowych, dla których będą pobierane dane.\\
Wynikiem analizy danych są wykresy zależności różnych czynników pogodowych od siebie, wykresy porównawcze dla różnych stacji pogodowych oraz automatycznie generowany komentarz dotyczący warunków w pobliżu danej stacji.

\subsection{Tabele}
\begin{itemize}
\small
\item Stacja pogodowa (id: //NazwaStacji//)
	\begin{itemize}
	\item Data
	\item Godzina
	\item Temperatury
		\begin{itemize}
		\item minimalna
		\item średnia
		\item maksymalna
		\end{itemize}
	\item Ciśnienie
	\item Wilgotność
	\item Wiatr
		\begin{itemize}
		\item kierunek
		\item prędkość
		\end{itemize}
	\item Opady
		\begin{itemize}
		\item deszcz
		\item śnieg
		\end{itemize}
	\item Zachmurzenie
	\end{itemize}
\item Stacje pogodowe (id: Stacje)
	\begin{itemize}
	\item Nazwa Stacji
	\item Lokalizacja
	\item Ostatnia aktualizacja
		\begin{itemize}
		\item Data
		\item Godzina
		\end{itemize}
	\item Średnia temperatura dzienna
	\item Opady dzienne
	\item Średnie ciśnienie dzienne
	\item Automatyczny komentarz
	\item Komentarz analityka
	\end{itemize}
\item Administratorzy (id: Admin)
	\begin{itemize}
	\item Login
	\item Hasło
	\item Imię
	\item Nazwisko
	\item e-mail
	\end{itemize}
\item Analitycy (id: Analityk)
	\begin{itemize}
	\item Login
	\item Hasło
	\item Imię
	\item Nazwisko
	\item e-mail
	\end{itemize}
\end{itemize}

\subsection{Zależności}

\subsection{Lista funkcjonalności}
\begin{enumerate}
\item Użytkownik
	\begin{itemize}
	\item Wyświetlenie analizy danych
		\begin{itemize}
		\item Wykresy zależności czynników pogodowych
		\item Wykresy porównawcze warunków między stacjami
		\item Komentarz dotyczący warunków
		\end{itemize}
	\item Wyświetlenie danych pogodowych
		\begin{itemize}
		\item Wyświetlenie tabeli Stacje
		\item Wyświetlenie tabel Stacja Pogodowa
		\end{itemize}
	\end{itemize}
\item Administrator (dziedziczy po użytkowniku)
	\begin{itemize}
	\item Zarządzanie danymi pogodowymi
		\begin{itemize}
		\item Dodanie stacji pogodowej
		\item Usunięcie stacji pogodowej
		\item Zmiana stacji pogodowej
		\item Ręczna aktualizacja danych
		\end{itemize}
	\item Zarządzanie wynikami analizy danych
		\begin{itemize}
		\item Zmiana komentarza analityka
		\item Ręczna aktualizacja wyników
		\end{itemize}
	\end{itemize}
\item Analityk (dziedziczy po użytkowniku)
	\begin{itemize}
	\item Zarządzanie wynikami analizy danych
		\begin{itemize}
		\item Zmiana komentarza analityka
		\item Ręczna aktualizacja wyników
		\end{itemize}
	\end{itemize}
\end{enumerate}

\end{document}


%\begin{figure}[H]
%\centering
%\includegraphics[width=\textwidth]{figures/03.png}
%\caption{Implementacja metody siecznych \label{fig:03}}
%\end{figure}
