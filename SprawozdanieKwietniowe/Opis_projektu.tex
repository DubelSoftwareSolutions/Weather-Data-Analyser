\documentclass[12pt,a4paper]{article}
\usepackage{indentfirst}
\usepackage[utf8x]{inputenc}
\usepackage{ucs}
\usepackage[MeX]{polski}
\usepackage{fancyhdr}
\usepackage{amsmath}
\usepackage{amsfonts}
\usepackage{amssymb}
\usepackage{subfig}
%\usepackage{supertabular}
\usepackage{array}
\usepackage{tabularx}
\usepackage{hhline}
\usepackage{tabulary, booktabs}
\usepackage{float}
\usepackage{lscape}
\usepackage[table]{xcolor}
\pagestyle{fancy}
\usepackage{graphicx}
\usepackage{multirow}
\usepackage[final]{pdfpages}
\newenvironment{bottompar}{\par\vspace*{\fill}}{\clearpage}
 
\begin{document}
\clearpage
\thispagestyle{empty}
 
\begin{figure}[H]
\centering
\includegraphics[scale=1.3]{logo.png}
\end{figure}
 
\vspace{16pt}
 
\begin{center}
\textbf{\huge Bazy Danych (Projekt)}
 
\vspace{30pt}
 
\textbf{\LARGE Analiza danych pogodowych}
 
 
\vspace{22pt}
 
\LARGE Opis projektu
 
\end{center}
 
\vspace{20pt}
 
\begin{flushleft}
Autorzy: Dymitr Choroszczak 218627, Krzysztof Dąbek 218549\\
Kurs: Bazy danych (projekt)\\
Temat: Analiza danych\\
Prowadzący: dr hab. inż. Grzegorz Mzyk, prof. PWr\\
Termin zajęć: piątek 9:15\\
\end{flushleft} 
 
\newpage
 
\tableofcontents
 
\newpage
 
\section{Opis projektu}
 
\subsection{Koncept projektu}
\normalsize
Projekt jest realizowany w ramach kursu Bazy Danych na specjalności Robotyka (ARR), na kierunku Automatyka i Robotyka (AiR), na wydziale Elektroniki (EKA), na Politechnice Wrocławskiej.\par
Celem projektu jest stworzenie bazy danych przechowującej informacje pogodowe z różnych stacji, pobierane z serwera oraz zaimplementowanie metod analizy danych i wizualizacji wyników.\par
Dane do analizy zostaną pobrane z serwera ogimet.com dla wielu stacji pogodowych w Polsce i zapisane w tabelach.\par
Zostaną stworzone dwie główne tabele z danymi, jedna będzie przechowywać podstawowe informacje o wszystkich aktualizowanych stacjach, druga informacje o warunkach pogodowych w pobliżu danej stacji z kolejnych dni.
Standardowy użytkownik bazy ma uprawnienia jedynie do wyświetlania wyników analizy oraz danych pogodowych.\par
Administrator ma dodatkowo możliwość dodawania, usuwania i zmiany stacji pogodowych, dla których będą pobierane dane.\par
Wynikiem analizy danych są wykresy zależności różnych czynników pogodowych od siebie, wykresy porównawcze dla różnych stacji pogodowych, automatycznie generowany komentarz dotyczący warunków w pobliżu danej stacji oraz komentarz analityka.

\subsection{Wymagania}
\subsubsection{Wymagania funkcjonalne}
\begin{enumerate}
\item Codzienna aktualizacja danych pogodowych
\item Ręczna aktualizacja danych pogodowych
\item Generowanie wykresów analitycznych
\item Generowanie automatycznej analizy
\item Wprowadzanie, zmiana komentarza analityka
\item Wyświetlanie informacji o stacjach pogodowych
\item Wyświetlanie danych pogodowych stacji
\item Wyświetlanie wyników analizy pogody
\item Dodawanie, usuwanie, zmiana stacji pogodowych
\item Logowanie administratorów i analityków
\item Przeprowadzanie wszystkich powyższych operacji tylko w zakresie uprawnień klienta
\end{enumerate}
\subsubsection{Wymagania niefunkcjonalne}
\begin{enumerate}
\item Odporność na utratę danych
\item Bezpieczeństwo danych administratorów i analityków
\item Wydajność
\item Komunikacja z serwerem danych pogodowych
\end{enumerate}

\subsection{Tabele i zależności}

\begin{figure}
\includegraphics[width=\textwidth]{•}./figures/diagram_zwiazkow.png}
\end{figure}

\begin{footnotesize}
\begin{itemize}
\item Stacje pogodowe (Stations id)
    \begin{itemize}
	\item Stacja pogodowa (MeteoStation id)
	\item Położenie (Localization id)
    \item Automatyczny komentarz
    \item Komentarz analityka
    \end{itemize}
\item Stacja pogodowa (MeteoStation id)
    \begin{itemize}
    \item Data
    \item Godzina
    \item Temperatura (Temperature id)
    \item Wilgotność
    \item Wiatr (Wind id)
    \item Ciśnienie
    \item Opady
    \item Zachmurzenie (Clouds id)
    \item Widoczność
    \end{itemize}
\item Położenie (Localization id)
	\begin{itemize}
	\item Stacje pogodowe (Stations id)
	\item Szerokość geograficzna
    \item Długość geograficzna
    \item Wysokość (m n.p.m.)
	\end{itemize}
\item Temperatura (Temperature id)
    \begin{itemize}
    \item Stacja pogodowa (MeteoStation id)
    \item minimalna
    \item średnia
    \item maksymalna
    \end{itemize}
\item Wiatr (Wind id)
    \begin{itemize}
    \item Stacja pogodowa (MeteoStation id)
    \item kierunek
    \item prędkość
    \item porywy
    \end{itemize}
\item Zachmurzenie (Clouds id)
	\begin{itemize}
	\item Wskaźnik całkowity
	\item Wskaźnik dolny
	\end{itemize}
\item Administratorzy (Admin id)
    \begin{itemize}
    \item Login
    \item Hasło
    \item Imię
    \item Nazwisko
    \item e-mail
    \end{itemize}
\item Analitycy (Analyst id)
    \begin{itemize}
    \item Login
    \item Hasło
    \item Imię
    \item Nazwisko
    \item e-mail
    \end{itemize}
\end{itemize}
\end{footnotesize}
\newpage
\subsection{Lista funkcjonalności}
%tabelka z rolami uzytkownikow
\begin{table}[!htb]
\centering
\caption{Poziomy kompetencji klientów} 
  \resizebox{\textwidth}{!}{\begin{tabular}{ | c | c | c | c | c |}   
    \hline
    \textbf{Klient }& \textbf{Stacje }& \textbf{Stacja }& \textbf{Położenie }&\textbf{ Temperatura }\\ \hline
    \textbf{Administrator }& owner & owner & table.readonly & table.readonly \\ \hline
    \textbf{Analityk }& table.readonly & table.readonly & table.readonly & table.readonly \\     \hline
    \textbf{Użytkownik }& table.readonly & table.readonly & table.readonly & table.readonly \\ \hline
  \end{tabular}}
\end{table}

\begin{table}[!htb]
\centering 
  \resizebox{\textwidth}{!}{\begin{tabular}{ | c | c | c | c | c |}   
    \hline
    \textbf{Klient }&\textbf{ Wiatr }& \textbf{Zachmurzenie }& \textbf{Administratorzy }& \textbf{Analitycy}\\ \hline    
    \textbf{Administrator }& table.readonly & table.readonly & owner & owner  \\ \hline
    \textbf{Analityk }& table.readonly & table.readonly & - & - \\    \hline
    \textbf{Użytkownik }& table.readonly & table.readonly & - & - \\ \hline
  \end{tabular}}
\end{table}

Rolom przyporządkowanym do poszczególnych użytkowników przyporządkowano następujące przywileje:
\begin{itemize}
\item owner : INSERT, SELECT, UPDATE, CREATE, DELETE
\item table.readonly : SELECT
\end{itemize}

\begin{figure}[!htb]
\includegraphics[width=\textwidth]{./figures/DiagramPrzypadkowUzycia.png}
\caption{Diagram przypadków użycia}
\end{figure}
 
\begin{small}
\begin{enumerate}
\item Użytkownik
    \begin{itemize}
    \item Wyświetlenie analizy danych
        \begin{itemize}
        \item Wykresy zależności czynników pogodowych
        \item Wykresy porównawcze warunków między stacjami
        \item Komentarz automatyczny
        \item Komentarz analityka
        \end{itemize}
    \item Wyświetlenie danych pogodowych
        \begin{itemize}
        \item Wyświetlenie tabeli Stacje
        \item Wyświetlenie tabel Stacja Pogodowa
        \end{itemize}
    \end{itemize}
\item Administrator (dziedziczy po użytkowniku)
    \begin{itemize}
    \item Zarządzanie danymi pogodowymi
        \begin{itemize}
        \item Dodanie stacji pogodowej
        \item Usunięcie stacji pogodowej
        \item Zmiana stacji pogodowej
        \item Ręczna aktualizacja danych
        \end{itemize}
    \item Zarządzanie wynikami analizy danych
        \begin{itemize}
        \item Zmiana komentarza analityka
        \item Ręczna aktualizacja wyników
        \end{itemize}
    \end{itemize}
\item Analityk (dziedziczy po użytkowniku)
    \begin{itemize}
    \item Zarządzanie wynikami analizy danych
        \begin{itemize}
        \item Zmiana komentarza analityka
        \item Ręczna aktualizacja wyników
        \end{itemize}
    \end{itemize}
\end{enumerate}
 
\end{small}
\end{document}