\documentclass[12pt,a4paper]{article}
\usepackage{indentfirst}
\usepackage[utf8x]{inputenc}
\usepackage{ucs}
\usepackage[MeX]{polski}
\usepackage{fancyhdr}
\usepackage{amsmath}
\usepackage{amsfonts}
\usepackage{amssymb}
\usepackage{subfig}
%\usepackage{supertabular}
\usepackage{array}
\usepackage{tabularx}
\usepackage{hhline}
\usepackage{tabulary, booktabs}
\usepackage{float}
\usepackage{lscape}
\usepackage[table]{xcolor}
\pagestyle{fancy}
\usepackage{graphicx}
\usepackage{multirow}
\newenvironment{bottompar}{\par\vspace*{\fill}}{\clearpage}

\begin{document}

\clearpage
\thispagestyle{empty}

\begin{figure}[H]
\centering
\includegraphics[scale=1.3]{logo.png}
\end{figure}

\vspace{16pt}

\begin{center}
\textbf{\huge Bazy Danych (Projekt)}

\vspace{30pt}

\textbf{\LARGE Analiza danych pogodowych}


\vspace{22pt}

\LARGE Opis projektu

\end{center}

\vspace{20pt}

\flushleft Autorzy: Dymitr Choroszczak 218627, Krzysztof Dąbek 218549\\
Kurs: Bazy danych (projekt)\\
Temat: Analiza danych\\
Prowadzący: dr hab. inż. Grzegorz Mzyk, prof. PWr\\
Termin zajęć: piątek 9:15\\


\newpage

\tableofcontents

\newpage

\section{Opis projektu}
\normalsize
Projekt jest realizowany w ramach kursu Bazy Danych na specjalności Robotyka (ARR), na kierunku Automatyka i Robotyka (AiR), na wydziale Elektroniki (EKA), na Politechnice Wrocławskiej.\\
Celem projektu jest stworzenie bazy danych przechowującej informacje pogodowe z różnych stacji, pobierane z serwera oraz zaimplementowanie metod analizy danych i wizualizacji wyników.\\


\subsection{Koncept projektu}

\subsection{Tabele}

\subsection{Zależności}

\subsection{Lista funkcjonalności}

\end{document}


%\begin{figure}[H]
%\centering
%\includegraphics[width=\textwidth]{figures/03.png}
%\caption{Implementacja metody siecznych \label{fig:03}}
%\end{figure}
